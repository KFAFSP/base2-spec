\documentclass[draft]{scrspec}

% Additional document macros.
% The MLIR arith dialect name.
\newcommand*{\arith}{\gls{mlir} \texttt{arith}\xspace}
% The base2 project name.
\newcommand*{\basetwo}{\textsc{base2}\xspace}
% Implementation-defined specifier.
\newcommand*{\impldef}{\emph{impl.-def.}\xspace}
% The LLVM project name.
\newcommand*{\LLVM}{\gls{llvm}\xspace}
% LLVMIR name.
\newcommand*{\LLVMIR}{\gls{llvmir}\xspace}
% The MLIR project name.
\newcommand*{\MLIR}{\gls{mlir}\xspace}


% Make and populate glossaries.
\newglossary[nlg]{notation}{not}{ntn}{Notation}
\loadglsentries[acronym]{content/acronyms}
\loadglsentries[notation]{content/notation}
\makeglossaries

\begin{document}

\title{\basetwo Specification}
\subtitle{Working Draft\\Version 0.5}

\author{
    Karl F. A. Friebel \\
    Technische Universität Dresden \\
    \texttt{karl.friebel@tu-dresden.de} \\
    \\
    Jiahong Bi \\
    Technische Universität Dresden \\
    \texttt{jiahong.bi@mailbox.tu-dresden.de} \\
    \vspace{20em}
}

\maketitle

% Revision History.
\begin{versionhistory}
    \vhEntry{0.1}{2023-01-20}{KF|JB}{created from markdown}
    \vhEntry{0.2}{2023-02-01}{KF}{fix errata in rounding definition}
    \vhEntry{0.3}{2023-03-01}{KF}{add type hierarchy and background}
    \vhEntry{0.4}{2023-03-02}{KF|JB}{release candidate 1}
    \vhEntry{0.5}{2023-03-05}{KF}{release candidate 2}
\end{versionhistory}

% TOC.
\newpage
\pdfbookmark[0]{\contentsname}{toc}
\tableofcontents

%
% Introduction and Overview
%

\part{Part 1: Introduction and Overview}

% Information about the document, acronyms, notation and references.
\chapter{About this Document}
\section{Scope of this Document}

% Document summary.
This document is the specification for the \basetwo \gls{ir}.
It defines the concepts, the type system and the operational semantics of the \basetwo \gls{ir}.
Additionally, it documents the \basetwo reference implementation based on the \MLIR framework.

% IR summary.
\basetwo is an \gls{ir} that introduces a common level of abstraction for operations on binary number types.
It focuses on reducing these programs to integer computations for targeting various devices.
\basetwo establishes a common baseline among \glspl{isa} of such devices, and provides a foundation for mapping them to reconfigurable compute targets.
It only reasons about binary fixed-point and floating-point numbers.

% Implementation summary.
The \MLIR implementation of \basetwo is the reference implementation of this standard.
It implements the \texttt{base2} dialect, which consists of types, attributes and operations that model the elements described in this specification.
The \texttt{base2} dialect also implements rewrites within the dialect itself, and translations to other targets, such as \LLVMIR.

% Author summary.
\basetwo was created at the \href{https://cfaed.tu-dresden.de/ccc-about}{Chair for Compiler Construction} at \gls{tud}.
It is released under the ISC license.

\begin{remark}{Note}
    This document is a working draft, and is subject to change.
    When proposing changes to the contents, please check that you are basing them on the newest available version.
\end{remark}

\section{Document Structure}

% Document outline.
This document is divided into the following parts:
\begin{itemize}
    \item Part 1 provides an overview over the \basetwo \gls{ir} design, along with some technical background on existing technologies and conventions.
    \item Part 2 is the specification of the \basetwo \gls{ir}.
    \item Part 3 is the documentation of the \texttt{base2} dialect implementation.
\end{itemize}

\glsaddall

% Acronyms.
\PrintGlossary{acronym}

% Notation.
\PrintGlossary{notation}

This document follows the requirements terminology outlined in \cite{RFC_2119}.

% References.
\PrintReferences{base2-spec}


% Background on existing technologies and conventions.
\chapter{Background}
\section{\glsentrytext{ssa} \glsentrytext{ir}}

% Summary.
A \gls{ssa} \gls{ir} is a form of program representation used by compiler infrastructures such as \LLVM.
For the purposes of this document, we use the following simplified definitions for the structure of a \gls{ssa} \gls{ir}.

\subsection{Elements}

% Elements.
Expressions in a \gls{ssa} \gls{ir} transform \term{values}[value] using \term{operations}[operation].
The \gls{ir} is statically-typed, meaning that every \term{value} has a compile-time constant \term{type}.

\begin{definition}[Type]
    A \newterm{type} \Type[T] is a uniquely identified set.
    The elements of this set are called the \newterm{instances}[instance] of \Type[T].
    \term{Instances}[instance] may be shared among \term{types}[type].
\end{definition}

\begin{definition}[Value]
    A \newterm{value} is a tuple \Value[T][t], holding an \term{instance} \(t \in \Type[T]\) of \term{type} \Type[T].
\end{definition}

\begin{definition}[Attribute]
    An \newterm{attribute} is a constant \term{value}.
\end{definition}

\begin{definition}[Operation]
    An \newterm{operation} \Op{} is a function between \term{values}[value]
    \begin{equation*}
        \Op{} : \Type[T]_1, \ldots, \Type[T]_n \to \Type[U]_1, \ldots, \Type[U]_m
    \end{equation*}
    where \(n\) is its \newterm{arity}.
\end{definition}

\begin{remark}{Note}
    All \term{operations}[operation] in \basetwo are \newterm{pure}[purity], i.e., may not observe or manipulate any implicit (machine) state.
\end{remark}

\subsection{Programs}

% Programs.
Programs in \gls{ssa} form use an unbounded number of \newterm{virtual registers}[register!virtual], which are uniquely-identified storage slots for \term{values}[value].
An algorithm in \gls{ssa} form is encoded using \term{assignments}[assignment] to these \term{virtual registers}[register!vitrual], and control flow.

\begin{definition}[Assignment]
    An \newterm{assignment} is the single \newterm{definition} of a tuple of uniquely identified \term{values}[value] \(d_i\)
    \begin{equation*}
        d_1, \ldots, d_m \gets \Op(u_1, \ldots, u_n)
    \end{equation*}
    where the \term{values}[value] \(u_i\) are the \newterm{uses}[use] of the \term{assignment}, and \Op{} is some \term{operation}.
\end{definition}

\begin{definition}[Program]
    A \newterm{program} is a \gls{dag} of \term{assignments}[assignment], connecting their \term{uses}[use] to their respective \term{definitions}[definition].
\end{definition}

\begin{remark}{Note}
    \basetwo does not model control flow, i.e., a \basetwo \term{program} is equivalent to an unterminated basic block in common terminology.
\end{remark}

\subsection{Undefined Behavior}

% Undefined behavior.
An \term{operation} is syntactically valid iff the \term{types}[type] of the involved \term{values}[value] match its definition.
However, the semantics of an \term{operation} may be undefined over specific \term{instances}[instance] of these \term{types}[type].
Evaluating an \term{operation} over \term{instances}[instance] it is not defined over is \gls{ub}.

\begin{definition}[Poison value]
    The \newterm{poison value}[value!poison] \Poison{} is a reserved symbol that represents the result of \gls{ub}.
    We allow its use both as a \term{value} and as an \term{instance} of any \term{type}.
\end{definition}

\section{Rounding}

% Summary.
\newterm{Rounding} is the process of representing an element of a set using its most accurate counterpart from another set.
The semantics of \term{rounding} are thus contextual.
Assuming the elements are partially ordered, such as number-like objects, we use \term{rounding functions}[rounding function] to disambiguate this.

\begin{definition}[Rounding function]
    A function \(\RoundR{} : X \to Y\) is a \newterm{rounding function} iff
    \begin{gather*}
        \label{eq:exact}
        \tag{exact}
        x \in X \cap Y \implies \RoundR{}(x) = x \\
        \label{eq:accurate}
        \tag{accurate}
        \RoundR{}(x) = x^\prime \implies R(x,x^\prime)
    \end{gather*}
    It must be exact, i.e., it is a superset of the identity over \(X \cap Y\).
    It must be accurate, i.e., it always upholds some accuracy guarantee given by a predicate \(R\).
\end{definition}

\subsection{Rounding Modes}

% Rounding modes.
\term{Rounding}[rounding] is often described in terms of an algorithm, also called a \newterm{rounding mode}.
\Cref{tab:rounding_modes} lists all \term{rounding modes}[rounding mode] used in this document, defined by their accuracy guarantee.

\begin{table}[H]
    \centering
    \begin{tabular}{|r|l|l|}
        \hline\rowcolor{tableheader}
        \(R\) & \bfseries Name & \bfseries Constraint \\ \hline
        \(\bot\)    & none           & --- \\
        \Exact      & exact          & \(x^\prime = x\) \\ \hline
                    &                & \(\min \lvert x^\prime-x \rvert\), such that \\
        \Nearest    & nearest        & --- \\
        \Down       & down           & \(x^\prime \ngtr x\)\\
        \Up         & up             & \(x^\prime \nless x\)\\
        \ToZero     & towards zero   & \(\lvert x^\prime\rvert \ngtr \lvert x\rvert\) \\
        \FromZero   & away from zero & \(\lvert x^\prime\rvert \nless \lvert x\rvert\) \\
        \Converge   & converge       & \(\underset{i\rightarrow\infty}{\lim}(x^\prime_i-x_i) = 0\) \\ \hline
    \end{tabular}
    \caption{Rounding modes}
    \label{tab:rounding_modes}
\end{table}

% Poison value.
Should the constraint be unsatisfiable for a given \(x\), \term{rounding} is \gls{ub}
\begin{equation*}
    \nexists y \in Y \ldotp R(x, y) \implies \RoundR{}(x) = \Poison
\end{equation*}

\subsection{Overflow and Underflow}

% Overflow and underflow.
Assuming the result set is bounded, a special case of \term{rounding} occurs when \(x > \max Y\) or \(x < \min Y\).
This is called \newterm{overflow} and \newterm{underflow} respectively.
The \term{rounding modes}[rounding mode] dictate a particular \newterm{saturation} behavior in this case, shown in \cref{tab:saturation}.

\begin{table}[H]
    \centering
    \begin{tabular}{|r|cc|r|}
        \hline\rowcolor{tableheader}
        \bfseries Mode & \bfseries Underflow & \bfseries Overflow & \bfseries Action \\ \hline
        \(\bot\)  & \multicolumn{2}{c|}{\impldef} & --- \\
        \Exact    & \Poison    & \Poison    & \emph{poison} \\ \hline
        \Nearest  & \(\min Y\) & \(\max Y\) & \emph{saturate} \\
        \Down     & \impldef   & \(\max Y\) & \emph{half-saturate} \\
        \Up       & \(\min Y\) & \impldef   & \emph{half-saturate} \\
        \ToZero   & \(\min Y\) & \(\max Y\) & \emph{saturate} \\
        \FromZero & \multicolumn{2}{c|}{\impldef} & --- \\
        \Converge & \multicolumn{2}{c|}{\impldef} & --- \\ \hline
    \end{tabular}
    \caption{Saturation on overflow/underflow}
    \label{tab:saturation}
\end{table}

\subsection{Broadening and Narrowing}

% Broadening and narrowing.
\newterm{Broadening}[broadening] is rounding where \(X \subseteq Y\), which is trivial and always exact.
The opposite is \newterm{narrowing}, where \(Y \subset X\) and the result is potentially inexact.
An example of \term{narrowing} is the \newterm{integer rounding function} \(\ZRoundR : \Rat \to \Int\).

\section{Quantization}

% Definition.
\newterm{Quantization}[quantization] is the process of mapping numbers observed in an algorithm to numbers that can be represented on the target machine.
For each variable, a \term{quantization function} must be determined.

\begin{definition}[Quantization function]
    A \newterm{quantization function} is a function that maps numbers from one field to another.
    It is not required to be injective, but an inverse must be defined.
\end{definition}

\subsection{Uniform Affine Integer Quantization}

% Uniform affine integer quantization.
A common form of \term{quantization} is mapping rational values to integers using linear functions.

\begin{definition}[Uniform affine integer quantization]
    A \newterm{uniform affine integer quantization}[quantization!uniform affine integer] and its \term{quantization function} \(\Pi: \Rat \to \Int\) are characterized by a \newterm{scaling factor} \(\delta x\) and a \newterm{zero point} offset \(x_0\)
    \begin{align*}
        \label{eq:linear_quant}
        \tag{linear-quant}
        \Pi(q) = \ZRound\left(\frac{q}{\delta x} - x_0\right) && \Pi^{-1}(z) = z\delta x + x_0
    \end{align*}
    where \ZRound{} is an \term{integer rounding function}.
\end{definition}

\section{Integer Division}

% Summary.
In an \term{integer division}, the rational result of a division is \term{rounded}[rounding] to an integer value.

\begin{definition}[Integer division]
    The \newterm{integer division} operator \IDivR{} is defined as
    \begin{align*}
        \IDivR{} : \Rat \times \Rat \to \Int \\
        \label{eq:idiv}
        \tag{idiv}
        a \IDivR{} b = \ZRoundR\left(\frac{a}{b}\right)
    \end{align*}
    where \ZRoundR{} is the \term{integer rounding function} and \(R\) is the \term{rounding mode}.
\end{definition}

\subsection{Modulo Operator}

% Introduction.
Some programming languages define a \Rem{} operator, which is often referred to as the \newterm{modulo operator}.
Assuming \(a, b \in \Rat\), it is usually constrained via the equation
\begin{align*}
    a = zb + (a \Rem b) && z \in \Int
\end{align*}
which is ambiguous in the choice of \(z\).

% Via integer division.
The modulo operator is closely associated with \term{integer division} \IDiv{}, which can be used to refine this definition.
For example, the C++20 standard \cite{ISO_C_N4860} gives the following equation
\begin{equation}
    \label{eq:rem}
    \tag{rem}
    a = \left(a \IDiv b\right)b + (a \Rem b)
\end{equation}
which is ambiguous in the rounding applied by \IDiv{}.
In the C++ standard, this used to be \impldef, but had to satisfy \(a \ge 0 \land b \ge 0 \implies (a \Rem b) \ge 0\).

% With implicit rounding.
Languages like Fortran and Python solve this issue by having a specified implicit rounding mode.
For example, the Fortran 202x standard \cite{ISO_F_N2184} defines the \texttt{MOD} and \texttt{MODULO} functions.
The former applies integer truncation (\ToZero), the latter rounds down (\Down).
For compatibility reasons, the C++0x standard adopted the \ToZero{} convention.
Python, on the other hand, rounds down (\Down), which is consistent with its integer division operator \texttt{//}.

% Integer truncation convention.
\begin{highlight}{Integer Truncation Convention}
The choice of rounding is controversial amongst users and maintainers alike.
An important advantage of truncation is that it satisfies
\begin{align*}
    (-a) \IDivR[\ToZero] b = -(a \IDivR[\ToZero] b) = a \IDivR[\ToZero] (-b) &&
    (a \RemR[\ToZero] b) < 0 \iff a < 0
\end{align*}
First, this means that integer division is associative.
Secondly, a non-negative dividend results in a non-negative remainder, which is useful for index calculations (e.g., ring buffers, etc.).
\end{highlight}

\subsection{Remainder}

% Remainder.
Disambiguating the modulo operator, we define the \term{remainder} of \term{integer division} as follows.

\begin{definition}[Remainder]
    The \newterm{remainder} operator \RemR{} is defined as
    \begin{align*}
        \RemR{} : \Rat \times \Rat \to \Rat \\
        \label{eq:divrem}
        \tag{divrem}
        a \RemR{} b = a - (a \IDivR b)b
    \end{align*}
    where \(R\) is the \term{rounding mode} of the underlying \term{integer division}.
\end{definition}

\section{Bit Sequences}

% Summary.
A binary data encoding represents a datum using an ordered sequences of bits.
Without loss of generality, this document implies a canonical order of the bits in a sequence.

\begin{definition}[Bit sequence]
    A \newterm{bit sequence} \(b = (\Sbit, \ldots, b_n)\) of length \(\Bwidth(b) = n\) is an ordered sequence of bits \(b_i \in \{0, 1\}\).
    Bit \Sbit{} is defined to be the \gls{msb}, and bit \(b_n\) is the \gls{lsb}.
\end{definition}

\begin{definition}[Bit universe]
    The \newterm{bit universe} \(\Buniv_n = \{0,1\}^n\) is the set of all \term{bit sequences}[bit sequence] of length \(n\).
\end{definition}

% ISA.
A hardware implementation of a machine using binary data formats provides storage for \term{bit sequences}[bit sequence] in the form of registers and memory.
The corresponding \gls{isa} defines how the instructions interact with this storage, which may be subject to global and contextual limitations.

\subsection{Padding}

% Addressing granularity.
The length of the shortest addressable \term{bit sequence} is the \newterm{addressing granularity} of the system.
A typical value is \(8\) bits, which corresponds to a \newterm{byte} being the smallest addressable unit of data.
Sequences shorter than this length are subject to implicit \term{padding}.

\begin{definition}[Padding]
    \newterm{Padding}[padding] refers to extending the length of a \term{bit sequence} by concatenating bits.
\end{definition}

% User-defined types.
By definition, the value of the \term{padded}[padding] bits is irrelevant.
Although purely an artifact of interpretation, a user-defined type is also said to be \term{padded}[padding] if it contains unused \term{bit subsequences}[bit sequence].

\subsection{Alignment}

% Address alignment.
Operands may be subject to instruction-dependent contextual addressing restrictions, such as an \term{alignment} restriction.
This restriction states that allowed addresses must be multiples of some integer.

\begin{definition}[Alignment]
    An \newterm{alignment}[alignment] of \(A \in \Nat\setminus\{0\}\) is a contextual- or \term{type} requirement on an \term{instance} address \(a\)
    \begin{equation*}
        a \equiv 0 \mod A
    \end{equation*}
\end{definition}

% Alignment and padding.
Suppose contiguous elements of size \(s\) starting from base address \(b\) must be used with \term{alignment} \(A\).
This requirement is trivially fulfilled if the element type is \term{aligned}[alignment] and sized in multiples of \(A\)
\begin{equation*}
    b \equiv 0 \mod A \land s \equiv 0 \mod A
\end{equation*}
which can be always be achieved by \term{padding}.

\section{Binary Integers}

% Summary.
A \newterm{binary integer}[integer!binary] is a binary encoding of an integer value, i.e., a (typed) \term{bit sequence} that has an unambiguous interpretation as an integer.
For the purposes of \basetwo, we define a canonical encoding.

\subsection{Unsigned}

% Unsigned integers.
\term{Unsigned integers}[integer!unsigned] of bit width \(n\) are commonly implemented to model the cyclic group \(\Int_{2^n}\).
This also defines their expected operational semantics.

\begin{definition}[Unsigned integer]
    A \term{bit sequence} \(b \in \Buniv_n\) is interpreted as an \newterm{unsigned integer}[integer!unsigned] following the canonical bit order
    \begin{equation}
        \label{eq:uint}
        \tag{uint}
        \InterpT[uint](b) = \sum_{i = 1}^{n} b_i 2^{n-i}
    \end{equation}
    Its limits are
    \begin{equation}
        \label{eq:uint_range}
        \tag{uint-range}
        \InterpT[uint](\Buniv_n) = [0, 2^n) \subset \Nat
    \end{equation}
\end{definition}

\subsection{Signed}

% Negative values.
Encoding a negative value without a sign symbol requires a modified encoding.
For example:

\begin{tabularx}{\textwidth}{>{\bfseries}lX}
    Sign \& magnitude &

    The sign and absolute value of the number are stored.
    The range of positive and negative values is symmetric, and \(+0=-0\).\\

    One's complement &

    Negative numbers are the bitwise complement of their absolute value.
    The range of positive and negative values is symmetric, and \(+0=-0\).\\

    Two's complement &

    Negative numbers are detectable via the \gls{msb}.
    There is one more negative value than there are positive values.
    Only \(0\) has magnitude \(0\).
\end{tabularx}

% Two's complement.
The \term{two's complement} representation is a homomorphism for the \term{signed integers}[integer!signed] onto the cyclic group \(\Int_{2^n}\), which are the corresponding \term{unsigned integers}[integer!unsigned].
In particular
\begin{equation*}
    \label{eq:sint_uint}
    \tag{sint-uint}
    \InterpT[uint](b) \equiv \InterpT[sint](b) \mod 2^n
\end{equation*}

\begin{definition}[Signed integer]
    A \term{bit sequence} \(b \in \Buniv_n\) is interpreted as a \newterm{signed integer}[integer!signed] following the canonical bit order and using the \newterm{two's complement} representation
    \begin{equation}
        \label{eq:sint}
        \tag{sint}
        \InterpT[sint](b) = -\Sbit 2^{n-1} + \sum_{i=1}^{n} b_i 2^{n-i} = \begin{cases}
            \InterpT[uint](b) & \Sbit = 0 \\
            2^N - \InterpT[uint](b) - 2 & \Sbit = 1
        \end{cases}
    \end{equation}
    where \Sbit{} is called the \newterm{sign bit}.
    Its limits are
    \begin{equation}
        \label{eq:sint_range}
        \tag{sint-range}
        \InterpT[sint](\Buniv_n) = [-2^{n-1}, 2^{n-1}) \subset \Int
    \end{equation}
\end{definition}

% Two's complement dominance.
\begin{highlight}{Two's Complement Dominance}
Virtually all systems with \term{binary integers}[integer!binary] in use today use the \term{two's complement} representation.
This lead to the ISO C/C++ committe reaching a consensus about the use of \term{two's complement} in their programming languages.
After previously only requiring \term{unsigned integers}[integer!unsigned] to model \(\Int_{2^n}\), C11 and C++20 now additionally require \term{two's complement}, und thus the same behavior, for \term{signed integers}[integer!signed].
See \cite{ISO_C_N2218,ISO_C_P0907R4} for discussions on advantages and drawbacks.
\end{highlight}

\subsection{Signless}

% Signedness
Since a \term{bit sequence} can be interpreted as either a \term{signed}[integer!signed] or \term{unsigned}[integer!unsigned] integer, a \term{binary integer}[integer!binary] must have an associated \newterm{signedness} to disambiguate.
\Glspl{ir} like \LLVMIR use \newterm{signless integers}[integer!signless], which do not carry this information, and thus do not have an interpretation.

\subsection{Truncation}

% Truncation.
A \term{binary integer}[integer!binary] \(z\) can be \term{truncated}[integer truncation] to a shorter bit width \(n\).
If \(z\) is unsigned, the result will be exact iff \(z < 2^n\), otherwise it will be exact iff \(2^{n-1} \le z < 2^{n}\).
Otherwise, information is lost during truncation.

\begin{definition}[Integer truncation]
    A \term{binary integer}[integer!binary] \(z\) of bit width \(n\) is \term{truncated}[integer truncation] to \(z^\prime\) with bit width \(m \le n\) by removing the first \(n - m\) \glspl{msb} such that
    \begin{equation*}
        \label{eq:trunc}
        \tag{trunc}
        z^\prime \equiv z \mod 2^m
    \end{equation*}
\end{definition}

\subsection{Extension}

% Extension.
A \term{binary integer}[integer!binary] can be appropriately \term{extended}[integer extension] to a longer bit width.
The result is always exact.

\begin{definition}[Integer extension]
    A \term{binary integer}[integer!binary] \(z\) of bit width \(n\) is \term{extended}[integer extension] to \(z^\prime\) with bit width \(m \ge n\) by prepending \(m - n\) \glspl{msb} such that
    \begin{equation*}
        \label{eq:ext}
        \tag{ext}
        z^\prime = z
    \end{equation*}
    \begin{itemize}
        \item For \term{unsigned integers}[integer!unsigned], \((0)^{m-n}\) is prepended (\newterm{zero extension}).
        \item For \term{signed integers}[integer!signed], \((\Sbit)^{m-n}\) is prepended (\newterm{sign extension}).
    \end{itemize}
\end{definition}

\subsection{Shifting}

% Bit shifting.
The digits of a \term{binary integer}[integer!binary] can be shifted towards the left or right.
Bits are inserted or discarded as needed.
Shifting is the strength-reduced form of multiplication by powers of \(2\).

\begin{definition}[Left shift]
    The \newterm{left shift} of \term{binary integer}[integer!binary] \(z\) of bit width \(n\) by \(k \in \Nat\) places is given by
    \begin{equation*}
        \label{eq:shl}
        \tag{shl}
        z \mathbin{\texttt{<<}} k \equiv z \cdot 2^k \mod 2^n
    \end{equation*}
\end{definition}

% Left shift.
For an \term{unsigned integer}[integer!unsigned], the result of \term{left shifting}[left shift] by \(k\) will be exact iff \((\Sbit, \ldots, b_k) = (0)^k\).
For a \term{signed integer}[integer!signed], the result of \term{left shifting}[left shift] by \(k\) will be exact iff \((\Sbit, \ldots, b_{k+1}) = (\Sbit)^{k+1}\).

\begin{definition}[Right shift]
    The \newterm{right shift} of \term{binary integer}[integer!binary] \(z\) of bit width \(n\) by \(k \in \Nat\) places is given by
    \begin{equation*}
        \label{eq:shr}
        \tag{shr}
        z \mathbin{\texttt{>>}} k = \lfloor z \cdot 2^{-k} \rfloor
    \end{equation*}
\end{definition}

\subsection{Byte Order}

% Introduction.
Hardware implementations of \term{binary integers}[integer!binary] can internally arrange the digits in any order.
As long as the data stays on the system, the bit order inside the unit of \term{addressing granularity} is irrelevant.

% Byte order.
Suppose the machine has \term{byte}-wise \term{addressing granularity}.
A \term{binary integer}[integer!binary] wider than a \term{byte} must be split and \term{padded}[padding] into \term{bytes}[byte] to be stored or transmitted.
Assuming consecutivity, the order of these \term{bytes}[byte] is called the \newterm{endian}.
The most common ones are:

\begin{tabularx}{\textwidth}{>{\bfseries}lX}
    little-endian &

    \term{Bytes}[byte] are stored from \gls{lsb} to \gls{msb}.\\

    big-endian &

    \term{Bytes}[byte] are stored from \gls{msb} to \gls{lsb}.\\
\end{tabularx}

\section{Binary Rationals}

% Summary.
Encoding a rational value in binary without a separator symbol requires a specialized encoding.
This can be achieved by using a pair of integer values to represent the rational value.

\begin{definition}[Binary rational number]
    The \newterm{binary rational numbers}[number!binary rational] model the rational numbers using an integer \newterm{significand} \(z\) and \newterm{binary exponent} \(E\).
    \begin{equation*}
        \Rat = \{z \cdot 2^E : z, E \in \Int\}
    \end{equation*}
\end{definition}

\subsection{Fixed-point}

% Fixed-point.
In cases where the \term{binary exponent} can be inferred from the context, it may be omitted.
For \(E=0\), the binary \term{binary fixed-point numbers}[number!binary fixed-point] are equivalent to the \term{binary integers}[integer!binary].

\begin{definition}[Binary fixed-point number]
    A \newterm{binary fixed-point number}[number!binary fixed-point] is a \term{binary rational number}[number!binary rational] with an implied exponent \(E\).
\end{definition}

\subsection{Floating-point}

% Floating-point.
Encoding the \term{binary exponent} explicitly increases the value range at the cost of accuracy.

\begin{definition}[Binary floating-point number]
    A \newterm{binary floating-point number}[number!binary floating-point] is a \term{binary rational number}[number!binary rational] with an explicit, variable exponent \(E\).
    Implementations may also represent non-rational values.
\end{definition}

\subsection{Non-finite}

% Non-rational values.
\term{Binary floating-point}[number!binary floating-point] implementations such as IEEE-754 \cite{IEEE_754} also represent non-finite values.
A non-finite value is not in \Rat{}, and not necessarily ordered with other values.

% Infinity.
IEEE-754 defines the positive and negative infinities \(\pm\infty\) and orders them as follows
\begin{gather*}
    \forall q \in \Rat{}\ldotp -\infty < q < \infty \\
    \lvert\pm\infty\rvert = \lvert\pm\infty\rvert
\end{gather*}
which means their magnitude is greater than that of any finite value.
In practice, they are represented using out-of-range \term{exponent}[binary exponent] values.

% NaN.
IEEE-754 represents values resulting from \gls{ub} as \gls{nan} values.
Different values exists, differing in payload and signalling behavior.
They are unordered with all values.


% Design rationale of the base2 project.
\chapter{Design Rationale}
\section{Goals}

% Summary.
\basetwo is an \MLIR-friendly abstraction of native binary number formats.
It is a \gls{ssa} \gls{ir} that models arithmetic expressions on binary numbers.

\subsection{Native Type and Operation Support}

\begin{itemize}
    \item[\Goal{goal:native_type}] Native number formats of a wide range of compute targets shall be representable.

    \begin{highlight}{}
        \basetwo is intended to be the interface layer between a target-agnostic front-end and a target-aware back-end.
        It must support at least all number types known to \LLVMIR.
    \end{highlight}

    \item[\Goal{goal:native_ops}] Common natively supported instructions shall have directly equivalent mappings.

    \begin{highlight}{}
        A target-aware manipulator must be able to perform code selection on the abstraction level of \basetwo.
        A subset with a well-defined lowering to \LLVMIR must exist.
    \end{highlight}

    \item[\Goal{goal:reduced}] \term{Number types}[type!number] shall support a common, reduced instruction set.

    \begin{highlight}{}
        A target-agnostic front-end must be able to express computations without knowledge of the target architecture.
        Implementations must ensure a minimal feature set is supported either natively or through emulation.
    \end{highlight}

    \item[\Goal{goal:param_types}] \term{Types}[type] shall be parametric to provide first-class support for reconfigurable compute.

    \begin{highlight}{}
        Reconfigurable targets may use arbitrary precision number formats, which are generally not representable in \LLVMIR or \arith.
    \end{highlight}
\end{itemize}

\subsection{Strong Type Semantics}

\begin{enumerate}
    \item[\Goal{goal:interpretation}] \term{Number types}[type!number] shall have unambiguous \term{interpretations}[interpretation].

    \begin{highlight}{}
        \LLVMIR, and \arith by extension, do not define an interpretation for integer instances.
        Instead, they associate the required information with the instructions.
        This precludes type-independent operational semantics.
    \end{highlight}

    \item[\Goal{goal:bits}] \term{Type instances}[instance] shall be represented by \term{bit sequences}[bit sequence].

    \begin{highlight}{}
        This limits the scope of \basetwo to the most common data organization scheme and traditional technologies.
    \end{highlight}

    \item[\Goal{goal:untyped_ops}] \term{Operations}[operation] shall be defined using type-independent semantics.

    \begin{highlight}{}
        To enable truly portable and reusable parametric types for target-agnostic compilers, operational semantics must be concise and independent of the involved types.
        \LLVMIR and \arith do not provide this (cf. \cref{goal:interpretation}).
    \end{highlight}
\end{enumerate}

\subsection{Canonical \glsentrytext{ub}}

\begin{enumerate}
    \item[\Goal{goal:ub}] \term{Operations}[operation] shall have clearly delimited \gls{ub}.

    \begin{highlight}{}
        Some scenarios require compiler exploitation of \gls{ub} to achieve good performance.
        \Glspl{ir} such as \arith do not clearly define these boundaries (yet), preventing such aggressive optimizations.
    \end{highlight}

    \item[\Goal{goal:rounding}] \term{Operations}[operation] shall follow well-defined \term{rounding} guarantees.

    \begin{highlight}{}
        The behavior of \basetwo \term{programs}[program] must be reproducible across different implementations within deterministic accuracy bounds.
        Target-agnostic front-ends need guarantees independent of the target lowering.
    \end{highlight}

    \item[\Goal{goal:impldef}] \term{Operations}[operation] shall accommodate clearly separated \impldef behavior.

    \begin{highlight}{}
        An \gls{ir} that does not account for \impldef behavior is unsuitable for code selection.
        Since \basetwo is intended to be closed, extensibility must be facilitated through \impldef customization points.
    \end{highlight}
\end{enumerate}

\section{Non-Goals}

% Summary.
\basetwo is designed to be a lightweight abstraction.
This section highlights the three categories of goals that are explicitly excluded from \basetwo's design.

\subsection{Algebraic Optimization}
\label{goal:algebra}

% No number theory.
\basetwo does not include reasoning about number-theoretical aspects, including group theory.
The operational semantics defined in this document only require a minimal set of requirements to be followed by the implementation.

\begin{highlight}{}
    Algebraic optimizations should be accommodated on higher abstraction levels.
\end{highlight}

\subsection{Arbitrary Encodings}
\label{goal:bit_encoding}

% Binary encodings.
\basetwo is not designed to reason about arbitrary binary encodings of numbers.
It specifically targets regular encodings which meaningfully support the \term{operations}[operation] defined in this document.
In particular, \basetwo only deals with binary \term{fixed-point}[number!binary fixed-point] and \term{floating-point numbers}[number!binary floating-point].

\begin{highlight}{}
    This separation from architectural details leaves the freedoms required for targeting different devices.
    Any implementation that preserves the binary radix and implements the canonical behavior (natively or emulated) can be conformant.
\end{highlight}

\subsection{Quantization}
\label{goal:quantization}

% Incompatibility with base2.
\basetwo is not designed to represent or reason about \term{quantization functions}[quantization function].
The type system of \basetwo does not allow carrying the parametrization of these functions.

\begin{highlight}{}
    \basetwo assists in \term{quantization} by providing the implementation of the native \term{type}, i.e., the \newterm{storage type} for the quantized number.
    In \MLIR, \basetwo \term{types}[type] are intended to be usable as a basis for truly quantized types.
\end{highlight}

\section{Assumptions}

% Summary.
The design of \basetwo makes some assumptions about typical uses and targets that limit its applicability.
These assumptions are based on observations made on current trends in hardware and programming language design.

% List of assumptions.
\begin{itemize}
    \item[\Assume{asm:ints}] Integers are the most fundamental native types.

    \begin{highlight}{Integer Dominance}
        % Integers are the lowest level of abstraction.
        \Cref{asm:ints} establishes an abstraction hierarchy of types in \basetwo, placing integer \term{operations}[operation] at the most specific level.
        This further implies that all \term{operations}[operation] are reducible to integer \term{operations}[operation].
    \end{highlight}

    \item[\Assume{asm:base2}] Binary number representations are the most common.

    \begin{highlight}{Binary Dominance}
        % Bits and two's complement rule the computing world.
        \Cref{asm:base2} limits the scope of the \basetwo abstraction to numerals with radix \(2\).
        This is the prevailing method of representing numbers in electronic compute platforms.
        Deviating targets are excluded.
    \end{highlight}

    \item[\Assume{asm:twos_compl}] \term{Two's complement}[two's complement] is the one true representation for signed binary integers.

    \begin{highlight}{Two's Complement Dominance}
        % Dominance of two's complement.
        \Cref{asm:twos_compl} requires that \term{signed integers}[integer!signed] are represented using the \term{two's complement}.
        This is the prevailing method of representing negative integers in binary format.
        Deviating targets are excluded.
    \end{highlight}

    \item[\Assume{asm:fixed_float}] Computations are either \term{fixed-point}[number!binary fixed-point] or \term{floating-point}[number!binary floating-point].

    \begin{highlight}{Fixed-point and Floating-point Dominance}
        % Fixed- and floating-point limitations.
        \Cref{asm:fixed_float} establishes a simple classification of numbers.
        All numbers in \basetwo are \term{binary rational numbers}[number!binary rational].
        Non-rational values are allowed in some cases.
        Interval arithmetic and exponents of bases other than \(2\) are excluded.
    \end{highlight}
\end{itemize}

\section{Requirements}

The concept of a \term{number type}[type!number] in \basetwo is constrained by the following requirements

% List of requirements.
\begin{itemize}
    \item[\Require{req:bits}] \term{Instances}[instance] of \term{number types}[type!number] must be \term{bit sequences}[bit sequence] of a type-specific, fixed length.

    \begin{highlight}{}
        All \term{number types}[type!number] in \basetwo are finite.
        No formats with sub-bit storage granularity are allowed.
        Variable length encodings are prohibited.
    \end{highlight}

    \item[\Require{req:interpretation}] \term{Instances}[instance] of \term{number types}[type!number] must have an unambiguous \term{interpretation}.

    \begin{highlight}{}
        Any typed \term{bit sequence} must have an unambiguous \term{interpretation}.
        The binary encoding must be fixed, and no contextual deviations are allowed.
    \end{highlight}

    \item[\Require{req:representation}] \term{Number types}[type!number] must provide a \term{representation} for the rational numbers.

    \begin{highlight}{}
        All \term{number types}[type!number] in \basetwo model \Rat{}.
        The \term{type} fully determines this mapping, which is based on a known accuracy guarantee.
    \end{highlight}

    \item[\Require{req:ordering}] \term{Instances}[instance] of \term{number types}[type!number] must have a consistent \term{partial ordering}[partial order].

    \begin{highlight}{}
        All objects represented by \term{number types}[type!number], including non-rational ones, must be ordered or unordered.
        This ordering must hold under \term{interpretation}.
    \end{highlight}

    \item[\Require{req:closed_arithmetic}] \term{Number types}[type!number] must implement add, subtract, multiply, divide and remainder.

    \begin{highlight}{}
        These closed arithmetic operators are the reduced instruction set, from which we derive all other semantics.
    \end{highlight}
\end{itemize}


%
% Specification
%

\part{Part 2: Specification}

% Specific concepts and terminology.
\chapter{Concepts}
% Summary.
\basetwo declares some specialized terminology to simplify and disambiguate its specification.
These concepts are tied to \basetwo specifically, and the definitions in this chapter take precedence over any external to this document.

\section{Interpretation}

% Summary.
\newterm{Interpretation}[interpretation] is the act of determining the numeric value that is represented by a \term{typed}[type] \term{bit sequence}.
For this purpose, we associate each \term{interpretable type}[type!interpretable] with an \term{interpretation function} \(\Interp\).

\begin{definition}[Interpretation function]
    The \newterm{interpretation function} \InterpT[P] of type \Type[P] takes the form
    \begin{gather*}
        \InterpT[P] : \Type[P] \to \PointsT[P] \\
        \PointsT[P] \subseteq \Rat \cup \ExtT[P]
    \end{gather*}
    where \PointsT[P] is the set of \term{points}[point] represented by \Type[P], and \ExtT[P] are the \impldef \term{points}[point] of \Type[P].
\end{definition}

% Total interpretation function.
We can extend this function to be defined over \Buniv{} by mapping all \term{bit sequences}[bit sequence] outside \Type[P] to \Poison
\begin{equation*}
    \label{eq:tot_interp}
    \tag{tot-interp}
    \Interp^{\prime}_{\Type[P]}(b) = \begin{cases}
        \InterpT[P](b) & b \in \Type[P] \\
        \Poison & b \notin \Type[P]
    \end{cases}
\end{equation*}
creating a \newterm{total interpretation function}[interpretation function!total] \(\Interp^{\prime}\).

% Inverse interpretation function.
Additionally, we assume the existence of an  \newterm{inverse interpretation function}[interpretation function!inverse] \(\Interp^{-1}\) such that
\begin{equation*}
    \label{eq:inv_interp}
    \tag{inv-interp}
    \InterpT[P]\left(\Interp^{-1}_{\Type[P]}\left(p\right)\right) = p
\end{equation*}
which selects some \term{bit sequence} representing \(p\) for all \(p \in \PointsT[P]\).

\subsection{Points}

% Summary.
\term{Number types}[type!number] in \basetwo model algebraic fields, which are comprised of number-like objects.
We introduce the concept of \term{points}[point] of an \term{interpretable type}[type!interpretable] to relax from rational numbers to such objects.

\begin{definition}[Point]
    A \newterm{point} is an object represented by an \term{interpretable type}[type!interpretable].
    A \term{point} \(p\) is either \newterm{rational}[point!rational] (\(p \in \Rat\)), or \newterm{extended}[point!extended] (\(p \notin \Rat\)).
\end{definition}

\subsubsection{Rational Points}

% Rational numbers.
A \newterm{rational point}[point!rational] is simply a rational number.
The field of rational numbers \Rat{} as modeled by \term{binary rational numerals}[rational!binary] is the full extent of \basetwo's number system.
All operational semantics of \basetwo are specified with respect to \term{rational points}[point!rational].

\subsubsection{Extended Points}

% Extended points.
An \term{interpretable type}[type!interpretable] \Type[P] may have an \impldef set \ExtT[P]{} of \newterm{extended points}[point!extended].
These \term{points}[point] may not represent rational numbers (\(\ExtT[P]{} \cap \Rat = \emptyset\)), and all uses of them constitute \impldef behavior.
Examples of this are non-finite values such as described in \cref{sec:non_finite}.

\section{Representation}

% Summary.
\newterm{Representation}[representation] is the act of constructing a \term{typed}[type] \term{bit sequence} from a numeric value.
It is the opposite of \term{interpretation}.
For this purpose, we associate each \term{number type}[type!number] with a \term{point rounding function}.

\begin{definition}[Point rounding function]
    The \newterm{point rounding function} \RoundTR[N] of \term{type} \Type[N] is a \term{rounding function} over the rational numbers
    \begin{equation*}
        \RoundTR[N] : \Rat \to \PointsT[N]
    \end{equation*}
\end{definition}

% Representation using inverse interpretation.
Given \(q \in \Rat\), the \term{bit sequence} \(b\) representing \(q\) in \term{type} \Type[N] with \term{accuracy guarantee}[rounding mode] \(R\) is obtained via
\begin{equation*}
    \label{eq:repr}
    \tag{repr}
    b = \Interp^{-1}_{\Type[N]}\left(\RoundTR[N]\left(q\right)\right)
\end{equation*}
where \(\Interp^{-1}_{\Type[N]}\) is the \term{inverse interpertation function}[interpretation function!inverse].
Following this definition, rational numbers in \Type[N] are part of an implied congruence relation \CongTR[N]{}
\begin{equation*}
    \label{eq:cong}
    \tag{cong}
    q \CongTR[N] \RoundTR[N]\left(q\right)
\end{equation*}

\subsection{Point-based Rounding}

% Point-based rounding.
Suppose \(x \in \Rat\) shall be represented in \term{type} \Type[N].
Let \(A, B \in \PointsT[N]\) be a pair of adjacent \term{points}[point] such that \(A \le x \le B\) and \(\nexists C\in\PointsT[N]\ldotp A < C < B\).
\Cref{fig:point_rounding} then provides the definition of \RoundTR[N].

\begin{figure}[H]
    \centering
    \begin{tikzpicture}[scale=2]
        % Number line.
        \draw[-latex] (-0.5,0) -- (3.5,0);
        \draw[fill=black] (0,0) circle (2pt) node[below=5pt] {A};
        \draw (1.5,-3pt) -- (1.5,3pt);
        \draw[fill=black] (3,0) circle (2pt) node[below=5pt] {B};

        % ULP.
        \draw (0,-12pt) -- (0,-18pt);
        \draw (3,-12pt) -- (3,-18pt);
        \draw[latex-latex] (0,-15pt) -- (3,-15pt) node [pos=0.5, below=3pt] {\(1\,\ulp\)};

        % Exact.
        \draw (-0.1,0.5) rectangle ++(0.2,-0.25) node[pos=0.5] {A};
        \draw (2.9,0.5) rectangle ++(0.2,-0.25) node[pos=0.5] {B};
        \draw (0.1,0.5) rectangle ++(2.8,-0.25) node[pos=0.5] {\Poison};
        \node at (-1.25,0.38) {exact (\Exact)};
        \node at (4,0.38) {\(= 0\,\ulp\)};

        % None.
        \draw (-0.1,0.8) rectangle ++(0.2,-0.25) node[pos=0.5] {A};
        \draw (2.9,0.8) rectangle ++(0.2,-0.25) node[pos=0.5] {B};
        \draw[pattern=north east lines] (0.1,0.8) rectangle ++(2.8,-0.25);
        \node at (-1.25,0.68) {none (\(\bot\))};
        \node at (4,0.68) {--};

        \node at (1.45,1.275) {\impldef};
        \draw[-latex] (1.5,1.4) -- (1.5,1.75);
        \draw[-latex] (1.5,1.1) -- (1.5,0.8);

        % Nearest.
        \draw (-0.1,2) rectangle ++(1.5,-0.25) node[pos=0.5] {A};
        \draw (1.6,2) rectangle ++(1.5,-0.25) node[pos=0.5] {B};
        \draw[pattern=north east lines] (1.4,2) rectangle ++(0.2,-0.25);
        \node at (-1.25,1.88) {nearest (\Nearest)};
        \node at (4,1.88) {\(\le 0.5\,\ulp\)};

        % Down.
        \draw (-0.1,2.3) rectangle ++(3.0,-0.25) node[pos=0.5] {A};
        \draw (2.9,2.3) rectangle ++(0.2,-0.25) node[pos=0.5] {B};
        \node at (-1.25,2.18) {down (\Down)};
        \node at (4,2.18) {\(< 1\,\ulp\)};

        % Up.
        \draw (0.1,2.6) rectangle ++(3.0,-0.25) node[pos=0.5] {B};
        \draw (-0.1,2.6) rectangle ++(0.2,-0.25) node[pos=0.5] {A};
        \node at (-1.25,2.48) {up (\Up)};
        \node at (4,2.48) {\(< 1\,\ulp\)};

        % Header.
        \node at (1.5,2.9) {\bfseries Result};
        \node at (-1.25,2.9) {\bfseries Mode};
        \node at (4,2.9) {\bfseries Error};
    \end{tikzpicture}
    \caption{\term{Point}[point]-based \term{rounding}}\label{fig:point_rounding}
\end{figure}

% ULP error.
The distance between \(A\) and \(B\) is defined as \(1\) \glsfirst{ulp}, which is used to bound the error of \term{rounding}.
A result is said to be mathematically \newterm{correctly rounded}[rounding!correct] when the error is \(\le 0.5\,\ulp\).

% Towards and away from zero.
The towards zero (\ToZero) and away from zero (\FromZero) \term{rounding modes}[rounding mode] are zero-relative (cf. \cref{tab:zero_rel_rounding}).

\begin{table}[H]
    \centering
    \begin{tabular}{|r|l|l|}
        \hline\rowcolor{tableheader}
        Case      & towards zero (\ToZero) & away from zero (\FromZero) \\
        \hline
        \(x < 0\) & up (\Up)               & down (\Down) \\
        \(x > 0\) & down (\Down)           & up (\Up) \\
        \hline
    \end{tabular}
    \caption{Zero-relative rounding}
    \label{tab:zero_rel_rounding}
\end{table}

% Collisions with the zero point.
Note that there is no requirement \(0 \in \PointsT[N]\), i.e., zero is not a required \term{point} of all \term{number types}[type!number].
It follows that rounding \(0\) using a zero-relative \term{rounding mode} may be undefined.
We relax this to
\begin{equation*}
    0 \notin \PointsT[N] \land (R = \FromZero \lor R = \ToZero) \implies \RoundTR[N](0) \in \{A, B\}
\end{equation*}

\section{Relations}

% Summary.
The concept of \term{interpretation} allows the definition of intuitive relations on \term{types}[type] and their \term{values}[value].

\subsection{Subtype Relation}

% Subtype relation.
Using the \term{points}[point] of \term{interpretable types}[type!interpretable], we establish a relation that determines whether the value range of one \term{type} is included within another.
Such a \term{type} is a \term{subtype} of the other.

\begin{definition}[Subtype relation]
    The \newterm{subtype relation}[subtype] between \term{interpretable types}[type!interpretable] \Type[A] and \Type[B] is given by
    \begin{equation*}
        \label{eq:subtype}
        \tag{subtype}
        \Type[A] \SubTEq \Type[B] \iff \PointsT[A] \subseteq \PointsT[B]
    \end{equation*}
    where \PointsT{} are the \term{points}[point] of \term{type} \Type[T].
\end{definition}

\subsection{Type Promotion}

% Type promotion.
An \newterm{archetype} is a logical class of related \term{number types}[type!number], which are often different instances of some parametric \term{type} template.
\term{Type promotion}[type promotion] produces a related type that unites the value ranges of the specified types from the same \term{archetype}.

\begin{definition}[Type promotion]
    The \newterm{type promotion operator}[type promotion] \Promote{} obtains the minimal common supertype of \Type[A] and \Type[B]
    \begin{align*}
        \Type[A] \Promote \Type[B] = \Type[AB] &\implies \Type[A] \SubTEq \Type[AB] \land \Type[B] \SubTEq \Type[AB] \tag{prom-super}\label{eq:prom_super}\\
        &\implies \nexists \Type[C] \ne \Type[AB]\ldotp\Type[A] \SubTEq \Type[C] \land \Type[B] \SubTEq \Type[C] \land \Type[C]\SubTEq\Type[AB] \tag{prom-min}\label{eq:prom_min}
    \end{align*}
    where \Type[A], \Type[B], \Type[AB], \Type[C] are \term{interpretable types}[type!interpretable] of the same \term{archetype}.
\end{definition}

% Type promotion operator.
Unless the trivial solution \(\Type[A] \SubTEq \Type[B] \implies \Type[A] \Promote \Type[B] = \Type[B]\) can be used, the implementation of the \term{promotion operator}[type promotion] is specific to the \term{archetype}.
However, the following identities must always hold
\begin{align*}
    \Type[A] \Promote \Type[B] &= \Type[B] \Promote \Type[A] \tag{prom-commute}\label{eq:prom_commute}\\
    \Type[A] \Promote \Type[A] &= \Type[A] \tag{prom-idem}\label{eq:prom_idem}\\
    \Type[A] \Promote (\Type[B] \Promote \Type[C]) &= (\Type[A] \Promote \Type[B]) \Promote \Type[C] \tag{prom-assoc}\label{prom:prom_assoc}\\
\end{align*}

\subsection{Partial Ordering}

% Partial ordering.
Given that the \term{points}[point] of an \term{interpretable type}[type!interpretable] are partially ordered, we extend this relation to its \term{values}[value].

\begin{definition}[Partial ordering relation]
    The \newterm{partial ordering relation} \(\LessT\) over \term{values}[value] of \term{interpretable types}[type!interpretable] \(\Type[T], \Type[U]\) is defined as
    \begin{equation*}
        (t : \Type[T]) \LessT (u : \Type[U]) \iff \InterpT[T](t) \preceq \InterpT[U](u)
    \end{equation*}
\end{definition}

% Minima and maxima.
We define the \newterm{minimal}[point!minimal] and \newterm{maximal}[point!maximal] \term{points}[point] of type \Type{} via
\begin{align}
    \Type_{\min} = \min \PointsT &= \{t_{\min} : \forall x \in \PointsT : t_{\min} \leq x \land x \nleq t_{\min}\}\tag{minima}\label{eq:minima} \\
    \Type_{\max} = \max \PointsT &= \{t_{\max} : \forall x \in \PointsT : x \leq t_{\max} \land t_{\max} \nleq x\}\tag{maxima}\label{eq:maxima}
\end{align}


\chapter{Types}

\begin{figure}
    \centering
    \begin{tikzpicture}
        \tikzset{type/.style = {shape=rectangle,draw,minimum size=1.5em}}
        \tikzset{specialization/.style = {->,>=Triangle}}

        \node[type] (B) at (0,3) {Bit sequence type \Type[B]};
        \node[type] (P) at (0,2) {Interpretable type \Type[P]};
        \node[type] (N) at (0,1) {Number type \Type[N]};
        \node[type] (F) at (-2,0) {Fixed-point type \Type[F]};
        \node[type] (R) at (2,0) {Floating-point type \Type[R]};
        \node[type] (I) at (-2,-1) {Integer type \Type[I]};

        \draw[specialization] (I) to (F);
        \draw[specialization] (R) to (N);
        \draw[specialization] (F) to (N);
        \draw[specialization] (N) to (P);
        \draw[specialization] (P) to (B);

        \tikzset{annot/.style = {shape=rectangle}}

        \node[annot] at (6,3) {\BwidthT[B]};
        \node[annot] at (6,2) {\InterpT[P]};
        \node[annot] at (6,1) {\RoundTR[N]};
        \node[annot] at (6,0) {\(z, E\)};
    \end{tikzpicture}
    \caption{The \basetwo type hierarchy.}
    \label{fig:type_hierarchy}
\end{figure}


% TODO.

%
% MLIR Reference Implementation
%

\part{Part 3: \glsentrytext{mlir} Reference Implementation}

% TODO.

\end{document}
