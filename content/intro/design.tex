\section{Goals}

% Summary.
\basetwo is an \MLIR-friendly abstraction of native binary number formats.
It is a \gls{ssa} \gls{ir} that models arithmetic expressions on binary numbers.

\subsection{Native Type and Operation Support}

\begin{itemize}
    \item[\Goal{goal:native_type}] Native number formats of a wide range of compute targets shall be representable.

    \begin{highlight}{}
        \basetwo is intended to be the interface layer between a target-agnostic front-end and a target-aware back-end.
        It must support at least all number types known to \LLVMIR.
    \end{highlight}

    \item[\Goal{goal:native_ops}] Common natively supported instructions shall have directly equivalent mappings.

    \begin{highlight}{}
        A target-aware manipulator must be able to perform code selection on the abstraction level of \basetwo.
        A subset with a well-defined lowering to \LLVMIR must exist.
    \end{highlight}

    \item[\Goal{goal:reduced}] \term{Number types}[type!number] shall support a common, reduced instruction set.

    \begin{highlight}{}
        A target-agnostic front-end must be able to express computations without knowledge of the target architecture.
        Implementations must ensure a minimal feature set is supported either natively or through emulation.
    \end{highlight}

    \item[\Goal{goal:param_types}] \term{Types}[type] shall be parametric to provide first-class support for reconfigurable compute.

    \begin{highlight}{}
        Reconfigurable targets may use arbitrary precision number formats, which are generally not representable in \LLVMIR or \arith.
    \end{highlight}
\end{itemize}

\subsection{Strong Type Semantics}

\begin{enumerate}
    \item[\Goal{goal:interpretation}] \term{Number types}[type!number] shall have unambiguous \term{interpretations}[interpretation].

    \begin{highlight}{}
        \LLVMIR, and \arith by extension, do not define an interpretation for integer instances.
        Instead, they associate the required information with the instructions.
        This precludes type-independent operational semantics.
    \end{highlight}

    \item[\Goal{goal:bits}] \term{Type instances}[instance] shall be represented by \term{bit sequences}[bit sequence].

    \begin{highlight}{}
        This limits the scope of \basetwo to the most common data organization scheme and traditional technologies.
    \end{highlight}

    \item[\Goal{goal:untyped_ops}] \term{Operations}[operation] shall be defined using type-independent semantics.

    \begin{highlight}{}
        To enable truly portable and reusable parametric types for target-agnostic compilers, operational semantics must be concise and independent of the involved types.
        \LLVMIR and \arith do not provide this (cf. \cref{goal:interpretation}).
    \end{highlight}
\end{enumerate}

\subsection{Canonical \glsentrytext{ub}}

\begin{enumerate}
    \item[\Goal{goal:ub}] \term{Operations}[operation] shall have clearly delimited \gls{ub}.

    \begin{highlight}{}
        Some scenarios require compiler exploitation of \gls{ub} to achieve good performance.
        \Glspl{ir} such as \arith do not clearly define these boundaries (yet), preventing such aggressive optimizations.
    \end{highlight}

    \item[\Goal{goal:rounding}] \term{Operations}[operation] shall follow well-defined \term{rounding} guarantees.

    \begin{highlight}{}
        The behavior of \basetwo \term{programs}[program] must be reproducible across different implementations within deterministic accuracy bounds.
        Target-agnostic front-ends need guarantees independent of the target lowering.
    \end{highlight}

    \item[\Goal{goal:impldef}] \term{Operations}[operation] shall accommodate clearly separated \impldef behavior.

    \begin{highlight}{}
        An \gls{ir} that does not account for \impldef behavior is unsuitable for code selection.
        Since \basetwo is intended to be closed, extensibility must be facilitated through \impldef customization points.
    \end{highlight}
\end{enumerate}

\section{Non-Goals}

% Summary.
\basetwo is designed to be a lightweight abstraction.
This section highlights the three categories of goals that are explicitly excluded from \basetwo's design.

\subsection{Algebraic Optimization}
\label{goal:algebra}

% No number theory.
\basetwo does not include reasoning about number-theoretical aspects, including group theory.
The operational semantics defined in this document only require a minimal set of requirements to be followed by the implementation.

\begin{highlight}{}
    Algebraic optimizations should be accommodated on higher abstraction levels.
\end{highlight}

\subsection{Arbitrary Encodings}
\label{goal:bit_encoding}

% Binary encodings.
\basetwo is not designed to reason about arbitrary binary encodings of numbers.
It specifically targets regular encodings which meaningfully support the \term{operations}[operation] defined in this document.
In particular, \basetwo only deals with binary \term{fixed-point}[number!binary fixed-point] and \term{floating-point numbers}[number!binary floating-point].

\begin{highlight}{}
    This separation from architectural details leaves the freedoms required for targeting different devices.
    Any implementation that preserves the binary radix and implements the canonical behavior (natively or emulated) can be conformant.
\end{highlight}

\subsection{Quantization}
\label{goal:quantization}

% Incompatibility with base2.
\basetwo is not designed to represent or reason about \term{quantization functions}[quantization function].
The type system of \basetwo does not allow carrying the parametrization of these functions.

\begin{highlight}{}
    \basetwo assists in \term{quantization} by providing the implementation of the native \term{type}, i.e., the \newterm{storage type} for the quantized number.
    In \MLIR, \basetwo \term{types}[type] are intended to be usable as a basis for truly quantized types.
\end{highlight}

\section{Assumptions}

% Summary.
The design of \basetwo makes some assumptions about typical uses and targets that limit its applicability.
These assumptions are based on observations made on current trends in hardware and programming language design.

% List of assumptions.
\begin{itemize}
    \item[\Assume{asm:ints}] Integers are the most fundamental native types.

    \begin{highlight}{Integer Dominance}
        % Integers are the lowest level of abstraction.
        \Cref{asm:ints} establishes an abstraction hierarchy of types in \basetwo, placing integer \term{operations}[operation] at the most specific level.
        This further implies that all \term{operations}[operation] are reducible to integer \term{operations}[operation].
    \end{highlight}

    \item[\Assume{asm:base2}] Binary number representations are the most common.

    \begin{highlight}{Binary Dominance}
        % Bits and two's complement rule the computing world.
        \Cref{asm:base2} limits the scope of the \basetwo abstraction to numerals with radix \(2\).
        This is the prevailing method of representing numbers in electronic compute platforms.
        Deviating targets are excluded.
    \end{highlight}

    \item[\Assume{asm:twos_compl}] \term{Two's complement}[two's complement] is the one true representation for signed binary integers.

    \begin{highlight}{Two's Complement Dominance}
        % Dominance of two's complement.
        \Cref{asm:twos_compl} requires that \term{signed integers}[integer!signed] are represented using the \term{two's complement}.
        This is the prevailing method of representing negative integers in binary format.
        Deviating targets are excluded.
    \end{highlight}

    \item[\Assume{asm:fixed_float}] Computations are either \term{fixed-point}[number!binary fixed-point] or \term{floating-point}[number!binary floating-point].

    \begin{highlight}{Fixed-point and Floating-point Dominance}
        % Fixed- and floating-point limitations.
        \Cref{asm:fixed_float} establishes a simple classification of numbers.
        All numbers in \basetwo are \term{binary rational numerals}[rational!binary].
        Irrational values are allowed in some cases.
        Interval arithmetic and exponents of bases other than \(2\) are excluded.
    \end{highlight}
\end{itemize}

\section{Requirements}

The concept of a \term{number type}[type!number] in \basetwo is constrained by the following requirements

% List of requirements.
\begin{itemize}
    \item[\Require{req:bits}] \term{Instances}[instance] of \term{number types}[type!number] must be \term{bit sequences}[bit sequence] of a type-specific, fixed length.

    \begin{highlight}{}
        All \term{number types}[type!number] in \basetwo are finite.
        No formats with sub-bit storage granularity are allowed.
        Variable length encodings are prohibited.
    \end{highlight}

    \item[\Require{req:interpretation}] \term{Instances}[instance] of \term{number types}[type!number] must have an unambiguous \term{interpretation}.

    \begin{highlight}{}
        Any typed \term{bit sequence} must have an unambiguous \term{interpretation}.
        The binary encoding must be fixed, and no contextual deviations are allowed.
    \end{highlight}

    \item[\Require{req:representation}] \term{Number types}[type!number] must provide a \term{representation} for the rational numbers.

    \begin{highlight}{}
        All \term{number types}[type!number] in \basetwo model \Rat{}.
        The \term{type} fully determines this mapping, which is based on a known \term{accuracy guarantee}[rounding mode].
    \end{highlight}

    \item[\Require{req:ordering}] \term{Instances}[instance] of \term{number types}[type!number] must have a consistent \term{partial ordering}[partial order].

    \begin{highlight}{}
        All objects represented by \term{number types}[type!number], including non-rational ones, must be ordered or unordered.
        This ordering must hold under \term{interpretation}.
    \end{highlight}

    \item[\Require{req:closed_arithmetic}] \term{Number types}[type!number] must implement add, subtract, multiply, divide and remainder.

    \begin{highlight}{}
        These closed arithmetic operators are the reduced instruction set, from which we derive all other semantics.
    \end{highlight}
\end{itemize}
