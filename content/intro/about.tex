\section{Scope of this Document}

% Document summary.
This document is the specification for the \basetwo \gls{ir}.
It defines the concepts, the type system and the operational semantics of the \basetwo \gls{ir}.
Additionally, it documents the \basetwo reference implementation based on the \MLIR framework.

% IR summary.
\basetwo is an \gls{ir} that introduces a common level of abstraction for operations on binary number types.
It focuses on reducing these programs to integer computations for targeting various devices.
\basetwo establishes a common baseline among \glspl{isa} of such devices, and provides a foundation for mapping them to reconfigurable compute targets.
It only reasons about binary fixed-point and floating-point numbers.

% Implementation summary.
The \MLIR implementation of \basetwo is the reference implementation of this standard.
It implements the \texttt{base2} dialect, which consists of types, attributes and operations that model the elements described in this specification.
The \texttt{base2} dialect also implements rewrites within the dialect itself, and translations to other targets, such as \LLVMIR.

% Author summary.
\basetwo was created at the \href{https://cfaed.tu-dresden.de/ccc-about}{Chair for Compiler Construction} at \gls{tud}.
It is released under the ISC license.

\begin{remark}{Note}
    This document is a working draft, and is subject to change.
    When proposing changes to the contents, please check that you are basing them on the newest available version.
\end{remark}

\section{Document Structure}

% Document outline.
This document is divided into the following parts:
\begin{itemize}
    \item Part 1 provides an overview over the \basetwo \gls{ir} design, along with some technical background on existing technologies and conventions.
    \item Part 2 is the specification of the \basetwo \gls{ir}.
    \item Part 3 is the documentation of the \texttt{base2} dialect implementation.
\end{itemize}

\glsaddall

% Acronyms.
\PrintGlossary{acronym}

% Notation.
\PrintGlossary{notation}

This document follows the requirements terminology outlined in \cite{RFC_2119}.

% References.
\PrintReferences{base2-spec}
