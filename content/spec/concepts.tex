% Summary.
\basetwo declares some specialized terminology to simplify and disambiguate its specification.
These concepts are tied to \basetwo specifically, and the definitions in this chapter take precedence over any external to this document.

\section{Interpretation}

% Summary.
\newterm{Interpretation}[interpretation] is the act of determining the numeric value that is represented by a \term{typed}[type] \term{bit sequence}.
For this purpose, we associate each \term{interpretable type}[type!interpretable] with an \term{interpretation function} \(\Interp\).

\begin{definition}[Interpretation function]
    The \newterm{interpretation function} \InterpT[P] of type \Type[P] takes the form
    \begin{gather*}
        \InterpT[P] : \Type[P] \to \PointsT[P] \\
        \PointsT[P] \subseteq \Rat \cup \ExtT[P]
    \end{gather*}
    where \PointsT[P] is the set of \term{points}[point] represented by \Type[P], and \ExtT[P] are the \impldef \term{points}[point] of \Type[P].
\end{definition}

% Total interpretation function.
We can extend this function to be defined over \Buniv{} by mapping all \term{bit sequences}[bit sequence] outside \Type[P] to \Poison
\begin{equation*}
    \label{eq:tot_interp}
    \tag{tot-interp}
    \Interp^{\prime}_{\Type[P]}(b) = \begin{cases}
        \InterpT[P](b) & b \in \Type[P] \\
        \Poison & b \notin \Type[P]
    \end{cases}
\end{equation*}
creating a \newterm{total interpretation function}[interpretation function!total] \(\Interp^{\prime}\).

% Inverse interpretation function.
Additionally, we assume the existence of an  \newterm{inverse interpretation function}[interpretation function!inverse] \(\Interp^{-1}\) such that
\begin{equation*}
    \label{eq:inv_interp}
    \tag{inv-interp}
    \InterpT[P]\left(\Interp^{-1}_{\Type[P]}\left(p\right)\right) = p
\end{equation*}
which selects some \term{bit sequence} representing \(p\) for all \(p \in \PointsT[P]\).

\subsection{Points}

% Summary.
\term{Number types}[type!number] in \basetwo model algebraic fields, which are comprised of number-like objects.
We introduce the concept of \term{points}[point] of an \term{interpretable type}[type!interpretable] to relax from rational numbers to such objects.

\begin{definition}[Point]
    A \newterm{point} is an object represented by an \term{interpretable type}[type!interpretable].
    A \term{point} \(p\) is either \newterm{rational}[point!rational] (\(p \in \Rat\)), or \newterm{extended}[point!extended] (\(p \notin \Rat\)).
\end{definition}

\subsubsection{Rational Points}

% Rational numbers.
A \newterm{rational point}[point!rational] is simply a rational number.
The field of rational numbers \Rat{} as modeled by \term{binary rational numerals}[rational!binary] is the full extent of \basetwo's number system.
All operational semantics of \basetwo are specified with respect to \term{rational points}[point!rational].

\subsubsection{Extended Points}

% Extended points.
An \term{interpretable type}[type!interpretable] \Type[P] may have an \impldef set \ExtT[P]{} of \newterm{extended points}[point!extended].
These \term{points}[point] may not represent rational numbers (\(\ExtT[P]{} \cap \Rat = \emptyset\)), and all uses of them constitute \impldef behavior.
Examples of this are non-finite values such as described in \cref{sec:non_finite}.

\section{Representation}

% Summary.
\newterm{Representation}[representation] is the act of constructing a \term{typed}[type] \term{bit sequence} from a numeric value.
It is the opposite of \term{interpretation}.
For this purpose, we associate each \term{number type}[type!number] with a \term{point rounding function}.

\begin{definition}[Point rounding function]
    The \newterm{point rounding function} \RoundTR[N] of \term{type} \Type[N] is a \term{rounding function} over the rational numbers
    \begin{equation*}
        \RoundTR[N] : \Rat \to \PointsT[N]
    \end{equation*}
\end{definition}

% Representation using inverse interpretation.
Given \(q \in \Rat\), the \term{bit sequence} \(b\) representing \(q\) in \term{type} \Type[N] with \term{accuracy guarantee}[rounding mode] \(R\) is obtained via
\begin{equation*}
    \label{eq:repr}
    \tag{repr}
    b = \Interp^{-1}_{\Type[N]}\left(\RoundTR[N]\left(q\right)\right)
\end{equation*}
where \(\Interp^{-1}_{\Type[N]}\) is the \term{inverse interpertation function}[interpretation function!inverse].
Following this definition, rational numbers in \Type[N] are part of an implied congruence relation \CongTR[N]{}
\begin{equation*}
    \label{eq:cong}
    \tag{cong}
    q \CongTR[N] \RoundTR[N]\left(q\right)
\end{equation*}

\subsection{Point-based Rounding}

% Point-based rounding.
Suppose \(x \in \Rat\) shall be represented in \term{type} \Type[N].
Let \(A, B \in \PointsT[N]\) be a pair of adjacent \term{points}[point] such that \(A \le x \le B\) and \(\nexists C\in\PointsT[N]\ldotp A < C < B\).
\Cref{fig:point_rounding} then provides the definition of \RoundTR[N].

\begin{figure}[H]
    \centering
    \begin{tikzpicture}[scale=2]
        % Number line.
        \draw[-latex] (-0.5,0) -- (3.5,0);
        \draw[fill=black] (0,0) circle (2pt) node[below=5pt] {A};
        \draw (1.5,-3pt) -- (1.5,3pt);
        \draw[fill=black] (3,0) circle (2pt) node[below=5pt] {B};

        % ULP.
        \draw (0,-12pt) -- (0,-18pt);
        \draw (3,-12pt) -- (3,-18pt);
        \draw[latex-latex] (0,-15pt) -- (3,-15pt) node [pos=0.5, below=3pt] {\(1\,\ulp\)};

        % Exact.
        \draw (-0.1,0.5) rectangle ++(0.2,-0.25) node[pos=0.5] {A};
        \draw (2.9,0.5) rectangle ++(0.2,-0.25) node[pos=0.5] {B};
        \draw (0.1,0.5) rectangle ++(2.8,-0.25) node[pos=0.5] {\Poison};
        \node at (-1.25,0.38) {exact (\Exact)};
        \node at (4,0.38) {\(= 0\,\ulp\)};

        % None.
        \draw (-0.1,0.8) rectangle ++(0.2,-0.25) node[pos=0.5] {A};
        \draw (2.9,0.8) rectangle ++(0.2,-0.25) node[pos=0.5] {B};
        \draw[pattern=north east lines] (0.1,0.8) rectangle ++(2.8,-0.25);
        \node at (-1.25,0.68) {none (\(\bot\))};
        \node at (4,0.68) {--};

        \node at (1.45,1.275) {\impldef};
        \draw[-latex] (1.5,1.4) -- (1.5,1.75);
        \draw[-latex] (1.5,1.1) -- (1.5,0.8);

        % Nearest.
        \draw (-0.1,2) rectangle ++(1.5,-0.25) node[pos=0.5] {A};
        \draw (1.6,2) rectangle ++(1.5,-0.25) node[pos=0.5] {B};
        \draw[pattern=north east lines] (1.4,2) rectangle ++(0.2,-0.25);
        \node at (-1.25,1.88) {nearest (\Nearest)};
        \node at (4,1.88) {\(\le 0.5\,\ulp\)};

        % Down.
        \draw (-0.1,2.3) rectangle ++(3.0,-0.25) node[pos=0.5] {A};
        \draw (2.9,2.3) rectangle ++(0.2,-0.25) node[pos=0.5] {B};
        \node at (-1.25,2.18) {down (\Down)};
        \node at (4,2.18) {\(< 1\,\ulp\)};

        % Up.
        \draw (0.1,2.6) rectangle ++(3.0,-0.25) node[pos=0.5] {B};
        \draw (-0.1,2.6) rectangle ++(0.2,-0.25) node[pos=0.5] {A};
        \node at (-1.25,2.48) {up (\Up)};
        \node at (4,2.48) {\(< 1\,\ulp\)};

        % Header.
        \node at (1.5,2.9) {\bfseries Result};
        \node at (-1.25,2.9) {\bfseries Mode};
        \node at (4,2.9) {\bfseries Error};
    \end{tikzpicture}
    \caption{\term{Point}[point]-based \term{rounding}}\label{fig:point_rounding}
\end{figure}

% ULP error.
The distance between \(A\) and \(B\) is defined as \(1\) \glsfirst{ulp}, which is used to bound the error of \term{rounding}.
A result is said to be mathematically \newterm{correctly rounded}[rounding!correct] when the error is \(\le 0.5\,\ulp\).

% Towards and away from zero.
The towards zero (\ToZero) and away from zero (\FromZero) \term{rounding modes}[rounding mode] are zero-relative (cf. \cref{tab:zero_rel_rounding}).

\begin{table}[H]
    \centering
    \begin{tabular}{|r|l|l|}
        \hline\rowcolor{tableheader}
        Case      & towards zero (\ToZero) & away from zero (\FromZero) \\
        \hline
        \(x < 0\) & up (\Up)               & down (\Down) \\
        \(x > 0\) & down (\Down)           & up (\Up) \\
        \hline
    \end{tabular}
    \caption{Zero-relative rounding}
    \label{tab:zero_rel_rounding}
\end{table}

% Collisions with the zero point.
Note that there is no requirement \(0 \in \PointsT[N]\), i.e., zero is not a required \term{point} of all \term{number types}[type!number].
It follows that rounding \(0\) using a zero-relative \term{rounding mode} may be undefined.
We relax this to
\begin{equation*}
    0 \notin \PointsT[N] \land (R = \FromZero \lor R = \ToZero) \implies \RoundTR[N](0) \in \{A, B\}
\end{equation*}

\section{Relations}

% Summary.
The concept of \term{interpretation} allows the definition of intuitive relations on \term{types}[type] and their \term{values}[value].

\subsection{Subtype Relation}

% Subtype relation.
Using the \term{points}[point] of \term{interpretable types}[type!interpretable], we establish a relation that determines whether the value range of one \term{type} is included within another.
Such a \term{type} is a \term{subtype} of the other.

\begin{definition}[Subtype relation]
    The \newterm{subtype relation}[subtype] between \term{interpretable types}[type!interpretable] \Type[A] and \Type[B] is given by
    \begin{equation*}
        \label{eq:subtype}
        \tag{subtype}
        \Type[A] \SubTEq \Type[B] \iff \PointsT[A] \subseteq \PointsT[B]
    \end{equation*}
    where \PointsT{} are the \term{points}[point] of \term{type} \Type[T].
\end{definition}

\subsection{Type Promotion}

% Type promotion.
An \newterm{archetype} is a logical class of related \term{number types}[type!number], which are often different instances of some parametric \term{type} template.
\term{Type promotion}[type promotion] produces a related type that unites the value ranges of the specified types from the same \term{archetype}.

\begin{definition}[Type promotion]
    The \newterm{type promotion operator}[type promotion] \Promote{} obtains the minimal common supertype of \Type[A] and \Type[B]
    \begin{align*}
        \Type[A] \Promote \Type[B] = \Type[AB] &\implies \Type[A] \SubTEq \Type[AB] \land \Type[B] \SubTEq \Type[AB] \tag{prom-super}\label{eq:prom_super}\\
        &\implies \nexists \Type[C] \ne \Type[AB]\ldotp\Type[A] \SubTEq \Type[C] \land \Type[B] \SubTEq \Type[C] \land \Type[C]\SubTEq\Type[AB] \tag{prom-min}\label{eq:prom_min}
    \end{align*}
    where \Type[A], \Type[B], \Type[AB], \Type[C] are \term{interpretable types}[type!interpretable] of the same \term{archetype}.
\end{definition}

% Type promotion operator.
Unless the trivial solution \(\Type[A] \SubTEq \Type[B] \implies \Type[A] \Promote \Type[B] = \Type[B]\) can be used, the implementation of the \term{promotion operator}[type promotion] is specific to the \term{archetype}.
However, the following identities must always hold
\begin{align*}
    \Type[A] \Promote \Type[B] &= \Type[B] \Promote \Type[A] \tag{prom-commute}\label{eq:prom_commute}\\
    \Type[A] \Promote \Type[A] &= \Type[A] \tag{prom-idem}\label{eq:prom_idem}\\
    \Type[A] \Promote (\Type[B] \Promote \Type[C]) &= (\Type[A] \Promote \Type[B]) \Promote \Type[C] \tag{prom-assoc}\label{prom:prom_assoc}\\
\end{align*}

\subsection{Partial Ordering}

% Partial ordering.
Given that the \term{points}[point] of an \term{interpretable type}[type!interpretable] are partially ordered, we extend this relation to its \term{values}[value].

\begin{definition}[Partial ordering relation]
    The \newterm{partial ordering relation} \(\LessT\) over \term{values}[value] of \term{interpretable types}[type!interpretable] \(\Type[T], \Type[U]\) is defined as
    \begin{equation*}
        (t : \Type[T]) \LessT (u : \Type[U]) \iff \InterpT[T](t) \preceq \InterpT[U](u)
    \end{equation*}
\end{definition}

% Minima and maxima.
We define the \newterm{minimal}[point!minimal] and \newterm{maximal}[point!maximal] \term{points}[point] of type \Type{} via
\begin{align}
    \Type_{\min} = \min \PointsT &= \{t_{\min} : \forall x \in \PointsT : t_{\min} \leq x \land x \nleq t_{\min}\}\tag{minima}\label{eq:minima} \\
    \Type_{\max} = \max \PointsT &= \{t_{\max} : \forall x \in \PointsT : x \leq t_{\max} \land t_{\max} \nleq x\}\tag{maxima}\label{eq:maxima}
\end{align}
